\usepackage[toc,nopostdot]{glossaries}

% https://www.craftbeer.com/beer/beer-glossary

\newglossaryentry{strike water}
{
  name={strike water},
  description={The water that is added to the malted grains that then transforms into the mash.}
}

\newglossaryentry{aeration}
{
  name={aeration},
  description={The action of introducing air or oxygen to the wort (unfermented beer) at various stages of the brewing process. Proper aeration before primary fermentation is vital to yeast health and vigorous fermentation. Aeration after fermentation is complete can result in beer off-flavors, including cardboard or paper aromas due to oxidation.}
}

\newglossaryentry{ale yeast}
{
  name={ale yeast},
  description={Saccharomyces cerevisiae is a top fermenting yeast that ferments at warm temperatures (60--70 F) and generally produces more flavor compounds.}
}

\newglossaryentry{all extract beer}
{
  name={all extract beer},
  description={A beer made with malt extract as opposed to one made from barley malt or from a combination of malt extract and barley malt.}
}
 
\newglossaryentry{All-Malt Beer}
{
  name={All-Malt Beer},
  description={A beer made entirely from mashed barley malt and without the addition of adjuncts, sugars or additional fermentables.}
}

\newglossaryentry{Alpha Acid}
{
  name={Alpha Acid},
  description={One of two primary naturally occurring soft resins in hops (the other is Beta Acid). Alpha acids are converted during wort boiling to iso-alpha acids, which cause the majority of beer bitterness. During aging, alpha acids can oxidize (chemical change) and lessen in bitterness.}
}

\newglossaryentry{Adjunct}
{
  name=Adjunct,
  description={Any unmalted grain or other fermentable ingredient used in the brewing process. Adjuncts used are typically either rice or corn, and can also include honey, syrups, and numerous other sources of fermentable carbohydrates. They are common in mass produced light American lager-style beers.}
  }

\newglossaryentry{Acid Rest}
{
  name={Acid Rest},
  description={A step done early in the mash around 95F by traditional brewers to lower the pH of the mash.}
  }

\newglossaryentry{Apparent Attenuation}
{
  name={Apparent Attenuation},
  description={A simple measure of the extent of fermentation that wort has undergone in the process of becoming beer. Using gravity units (GU), Balling (B), or Plato (P) units to express gravity, apparent attenuation is equal to the original gravity minus the final gravity divided by the original gravity. The result is expressed as a percentage and equals 65\% to 80\% for most beers.}
  }

\newglossaryentry{Astringency}
{
  name={Astringency},
  description={A characteristic of beer taste mostly caused by tannins, oxidized (phenols), and various aldehydes (in stale beer). Astringency can cause the mouth to pucker and is often perceived as dryness.}
  }

\newglossaryentry{Attenuation}
{
  name={Attenuation},
  description={The reduction in wort specific gravity caused by the yeast consuming wort sugars and converting them into alcohol and carbon dioxide gas through fermentation.}
  }

\newglossaryentry{Autolysis}
{
  name={Autolysis},
  description={A process in which excess yeast cells feed on each other producing a rubbery or vegetal aroma.}
  }

\newglossaryentry{Blending}
{
  name={Blending},
  description={The mixing together of different batches of beer to create a final product.}
  }

\newglossaryentry{Body}
{
  name={Body},
  description={The consistency, thickness and mouth-filling property of a beer. The sensation of palate fullness in the mouth ranges from thin- to full-bodied.}
  }

\newglossaryentry{Boiling}
{
  name={Boiling},
  description={A critical step during the brewing process during which wort (unfermented beer) is boiled inside the brew kettle. During the boiling, one or more hop additions can occur to achieve bittering, hop flavor and hop aroma in the finished beer. Boiling also results in the removal of several volatile compounds from wort, especially dimethyl sulfide (see below) and the coagulation of excess or unwanted proteins in the wort (see ``hot break''). Boiling also sterilizes a beer as well as ends enzymatic conversion of proteins to sugars.}
  }

\newglossaryentry{Bottle Conditioning}
{
  name={Bottle Conditioning},
  description={A process by which beer is naturally carbonated in the bottle as a result of fermentation of additional wort or sugar intentionally added during packaging.}
  }

\newglossaryentry{Bottom Fermentation}
{
  name={Bottom Firmentation},
  description={One of the two basic fermentation methods characterized by the tendency of yeast cells to sink to the bottom of the fermentation vessel. Lager yeast is considered to be bottom fermenting compared to ale yeast that is top fermenting. Beers brewed in this fashion are commonly called lagers or bottom-fermented beers.}
  }

\newglossaryentry{Brew Kettle}
{
  name={Brew Kettle},
  description={One of the vessels used in the brewing process in which the wort (unfermented beer) is boiled.}
  }

\newglossaryentry{Bung}
{
  name={Bung},
  description={A sealing stopper, usually a cylindroconical shaped piece of wood or plastic, fitted into the mouth of a cask or older style kegs such as Hoff-Stevens or Golden Gate.}
  }

\newglossaryentry{Bung Hole}
{
  name={Bung Hole},
  description={The round hole in the side of a cask or older style keg, through which the vessel is filled with beer and then sealed with a bung.}
  }

\newglossaryentry{Byproducts}
{
  name={Byproducts},
  description={Desirable and undesirable compounds that are a result of fermentation, mashing, and boiling.}
  }

\newglossaryentry{Calcium Carbonate (CaCO3)}
{
  name={Calcium Carbonate (CaCO3)},
  description={A mineral common in water of different origins. Also known as chalk, sometimes added during brewing to increase calcium and carbonate content.}
  }

\newglossaryentry{Calcium Sulfate (CaSO4)}
{
  name={Calcium Sulfate (CaSO4)},
  description={A mineral common in water of different origins. Also known as gypsum, sometimes added during brewing to increase calcium and sulfate content.}
  }

\newglossaryentry{Carbohydrates}
{
  name={Carbohydrates},
  description={A group of organic compounds including sugars and starches, many of which are suitable as food for yeast and bacteria.}
  }

\newglossaryentry{Carbon Dioxide (CO2)}
{
  name={Carbon Dioxide (CO2)},
  description={The gaseous by-product of yeast. Carbon dioxide is what gives beer its carbonation (bubbles).}
  }

\newglossaryentry{Carbonation}
{
  name={Carbonation},
  description={The process of introducing carbon dioxide into a liquid (such as beer) by: pressurizing a fermentation vessel to capture naturally produced carbon dioxide; injecting the finished beer with carbon dioxide; adding young fermenting beer to finished beer for a renewed fermentation (kraeusening); priming (adding sugar to) fermented wort prior to packaging, creating a secondary fermentation in the bottle, also known as ``bottle conditioning.''}
  }
\newglossaryentry{Carboy}
{
  name={Carboy},
  description={A large glass, plastic or earthenware bottle.}
  }

\newglossaryentry{Caryophyllene}
{
  name={Caryophyllene},
  description={One of the essential oils made in the flowering cone of the hops plant Humulus lupulus.}
  }

\newglossaryentry{Cask}
{
  name={Cask},
  description={A barrel-shaped container for holding beer. Originally made of iron-hooped wooden staves, now most widely available in stainless steel and aluminum.}
  }

\newglossaryentry{Cask Conditioning}
{
  name={Cask Conditioning},
  description={Storing unpasteurized, unfiltered beer for several days in cool cellars of about \SI{48}-\SI{56}\celsius (\SI{13}\celsius) while conditioning is completed and carbonation builds.}
  }

\newglossaryentry{Cellaring}
{
  name={Cellaring},
  description={Storing or aging beer at a controlled temperature to allow maturing.}
}

\newglossaryentry{Chill Haze}
{
  name={Chill Haze},
  description={Hazy or cloudy appearance caused when the proteins and tannins naturally found in finished beer combine upon chilling into particles large enough to reflect light or become visible.}
  }

\newglossaryentry{Closed Fermentation}
{
  name={Closed Firmentation},
  description={Fermentation under closed, anaerobic conditions to minimize risk of contamination and oxidation.}
  }

\newglossaryentry{Cold Break}
{
  name={Cold Break},
  description={The flocculation of proteins and tannins during wort cooling.}
  }

\newglossaryentry{Color}
{
  name={Color},
  description={The hue or shade of a beer, primarily derived from grains, sometimes derived from fruit or other ingredients in beer. Beer styles made with caramelized, toasted or roasted malts or grains will exhibit increasingly darker colors. The color of a beer may often, but not always, allow the consumer to anticipate how a beer might taste. It’s important to note that beer color does not equate to alcohol level, mouthfeel or calories in beer.}
  }

\newglossaryentry{Conditioning}
{
  name={Conditioning},
  description={A step in the brewing process in which beer is matured or aged after initial fermentation to prevent the formation of unwanted flavors and compounds}
  }

\newglossaryentry{Contract Brewing Company}
{
  name={Contract Brewing Company},
  description={A business that hires another brewery to produce some or all of its beer. The contract brewing company handles marketing, sales and distribution of its beer, while generally leaving the brewing and packaging to its producer-brewery.}
  }

\newglossaryentry{Craft Brewery}
{
  name={Craft Brewery},
  description={According to the Brewers Association, an American craft brewer is small, independent and traditional.  \textbf{Small:} Annual production of 6 million barrels of beer or less (approximately 3 percent of U.S. annual sales). Beer production is attributed to the rules of alternating proprietorships.  \textbf{Independent:} Less than 25 percent of the craft brewery is owned or controlled (or equivalent economic interest) by a beverage alcohol industry member that is not itself a craft brewer.  \textbf{Traditional:} A brewer that has a majority of its total beverage alcohol volume in beers whose flavor derives from traditional or innovative brewing ingredients and their fermentation. Flavored malt beverages (FMBs) are not considered beers.  }
  }

\newglossaryentry{Decoction Mash}
{
  name={Decoction Mash},
  description={A method of mashing that raises the temperature of the mash by removing a portion, boiling it, and returning it to the mash tun. Often used multiple times in certain mash programs.}
  }

\newglossaryentry{Degrees Plato}
{
  name={Degrees Plato},
  description={An empirically derived hydrometer scale to measure density of beer wort in terms of percentage of extract by weight.}
  }

\newglossaryentry{Dextrin}
{
  name={Dextrin},
  description={A group of complex, unfermentable and tasteless carbohydrates produced by the partial hydrolysis of starch, that contributes to the gravity and body of beer. Some dextrins remain undissolved in the finished beer, giving it a malty sweetness.}
  }

\newglossaryentry{Diacetyl}
{
  name={Diacetyl},
  description={A volatile compound produced by some yeasts which imparts a caramel, nutty or butterscotch flavor to beer. This compound is acceptable at low levels in several traditional beer styles, including: English and Scottish Ales, Czech Pilsners and German Oktoberfest. However, it is often an unwanted or accidental off-flavor.}
  }

\newglossaryentry{Diastatic}
{
  name={Diastatic},
  description={Refers to the diastatic enzymes that are created as the grain sprouts. These convert starches to sugars, which yeast eat.}
  }

\newglossaryentry{Dimethyl Sulfide (DMS)}
{
  name={Dimethyl Sulfide (DMS)},
  description={At low levels, DMS can impart a favorable sweet aroma in beer. At higher levels, DMS can impart a characteristic aroma and taste of cooked vegetables, such as cooked corn or celery. Low levels are acceptable in and characteristic of some Lager beer styles.}
  }

\newglossaryentry{Draught Beer}
{
  name={Draught Beer},
  description={Beer drawn from kegs, casks or serving tanks rather than from cans, bottles or other packages. Beer consumed from a growler relatively soon after filling is also sometimes considered draught beer. Learn more: Draught Quality Manual.}
  }

\newglossaryentry{Dry Hopping}
{
  name={Dry Hopping},
  description={The addition of hops late in the brewing process to increase the hop aroma of a finished beer without significantly affecting its bitterness. Dry hops may be added to the wort in the kettle, whirlpool, hop back, or added to beer during primary or secondary fermentation or even later in the process.}
  }

\newglossaryentry{Dual Purpose Hops}
{
  name={Dual Purpose Hops},
  description={Hops that are added to provide both bittering and aromatic properties.}
  }

\newglossaryentry{Endosperm}
{
  name={Endosperm},
  description={The starch-containing sac of the barley grain.}
  }

\newglossaryentry{Essential Hop Oils}
{
  name={Essential Hop Oils},
  description={Essential hop oils are what is isomerized in wort and provide the aromatic and flavor compounds that are associated with hop additions.}
  }

\newglossaryentry{Esters}
{
  name={Esters},
  description={Volatile flavor compounds that form through the interaction of organic acids with alcohols during fermentation and contribute to the fruity aroma and flavor of beer. Esters are very common in ales.}
  }

\newglossaryentry{Ethanol}
{
  name={Ethanol},
  description={Ethyl alcohol, the colorless primary alcohol constituent of beer.}
  }

\newglossaryentry{Export}
{
  name={Export},
  description={Any beer produced for the express purpose of exportation. For example: export-style German lagers or export-style Irish stouts.}
  }

\newglossaryentry{Farnesene}
{
  name={Farnesene},
  description={One of the essential oils made in the flowering cone of the hops plant Humulus lupulus.}
  }

\newglossaryentry{Fermentable Sugars}
{
  name={Fermentable Sugars},
  description={Sugars that can be consumed by yeast cells which in turn will produce ethanol alcohol and c02.}
  }

\newglossaryentry{Fermentation}
{
  name={Fermentation},
  description={The chemical conversion of fermentable sugars into approximately equal parts of ethyl alcohol and carbon dioxide gas, through the action of yeast. The two basic methods of fermentation in brewing are top fermentation, which produces ales, and bottom fermentation, which produces lagers.}
  }

\newglossaryentry{Fermentation Lock}
{
  name={Fermentation Lock},
  description={A one-way valve, often made of glass or plastic that is fitted onto a fermenter and allows carbon dioxide gas to escape from the fermenter while excluding ambient wild yeasts, bacteria and contaminants.}
  }

\newglossaryentry{Filtration}
{
  name={Filtration},
  description={The passage of a liquid through a permeable or porous substance to remove solid matter in suspension, often yeast.}
  }

\newglossaryentry{Final Gravity}
{
  name={Final Gravity},
  description={The specific gravity of a beer as measured when fermentation is complete (when all desired fermentable sugars have been converted to alcohol and carbon dioxide gas). Synonym: Final specific gravity; final SG; finishing gravity; terminal gravity.}
  }

\newglossaryentry{Fining}
{
  name={Fining},
  description={The process of adding clarifying agents such as isinglass, gelatin, silica gel, or Polyvinyl Polypyrrolidone (PVPP) to beer during secondary fermentation to hasten the precipitation of suspended matter, such as yeast, proteins or tannins.}
  }

\newglossaryentry{Flocculation}
{
  name={Flocculation},
  description={The behavior of suspended particles in wort or beer that tend to clump together in large masses and settle out. During brewing, protein and tannin particles will flocculate out of the kettle, coolship or fermenter during hot or cold break. During and at the end of fermentation, yeast cells will flocculate to varying degrees depending on the yeast strain, thereby affecting fermentation as well as filtration of the resulting beer.}
  }

\newglossaryentry{Forced Carbonation}
{
  name={Forced Carbonation},
  description={The beer is placed into a sealed (or soon to be sealed) container and carbonation is rapidly added. Under high pressure, the CO2 is absorbed into the beer.}
  }

\newglossaryentry{Fresh Hopping}
{
  name={Fresh Hopping},
  description={The addition of freshly harvested hops that have not yet been dried to different stages of the brewing process. Fresh hopping adds unique flavors and aromas to beer that are not normally found when using hops that have been dried and processed per usual. Synonymous with wet hopping.}
  }

\newglossaryentry{Fusel Alcohol}
{
  name={Fusel Alcohol},
  description={A family of high molecular weight alcohols, which result from excessively high fermentation temperatures. Fusel alcohols can impart harsh or solvent-like characteristics commonly described as lacquer or paint thinner. It can contribute to hangovers.}
  }

\newglossaryentry{Germination}
{
  name={Germination},
  description={Growth of a barley grain as it produces a rootlet and acrospire.}
  }

\newglossaryentry{Grainy}
{
  name={Grainy},
  description={Tasting or smelling like cereal or raw grains.}
  }

\newglossaryentry{Grist}
{
  name={Grist},
  description={Ground malt and grains ready for mashing.}
  }

\newglossaryentry{Growler}
{
  name={Growler},
  description={A jug- or pail-like container once used to carry draught beer bought by the measure at the local tavern. Growlers are usually $\frac{1}{2}$ gal (64 oz) or 2L (68 oz) in volume and made of glass. Brewpubs often serve growlers to sell beer to-go. Often a customer will pay a deposit on the growler but can bring it back again and again for a re-fill. Growlers to-go are not legal in all U.S. states.}
  }

\newglossaryentry{Gruit}
{
  name={Gruit},
  description={An old-fashioned herb mixture used for bittering and flavoring beer, popular before the extensive use of hops. Gruit or grut ale may also refer to the beverage produced using gruit.}
  }

\newglossaryentry{Hand Pump}
{
  name={Hand Pump},
  description={A device for dispensing cask conditioned draught beer using a pump operated by hand. The use of a hand pump allows the draught beer to be served without the use of pressurized carbon dioxide.}
  }

\newglossaryentry{Head Retention}
{
  name={Head Retention},
  description={The foam stability of a beer as measured, in seconds, by time required for a 1-inch foam collar to collapse.}
  }

\newglossaryentry{Heat Exchangers}
{
  name={Heat Exchangers},
  description={Used to cool hot wort before fermentation.}
  }

\newglossaryentry{Homebrewing}
{
  name={Homebrewing},
  description={The art of making beer at home. In the U.S., homebrewing was legalized by President Carter on February 1, 1979, through a bill introduced by California Senator Alan Cranston. The Cranston Bill allows a single person to brew up to 100 gallons of beer annually for personal enjoyment and up to 200 gallons in a household of two persons or more of legal drinking age. Learn more from the American Homebrewers Association.}
  }

\newglossaryentry{Hops}
{
  name={Hops},
  description={A perennial climbing vine, also known by the Latin botanical name Humulus lupulus. The female plant yields flowers of soft-leaved pine-like cones (strobile) measuring about an inch in length. Only the female ripened flower is used for flavoring beer. Because hops reproduce through cuttings, the male plants are not cultivated and are even rooted out to prevent them from fertilizing the female plants, as the cones would become weighed-down with seeds. Seedless hops have a much higher bittering power than seeded. There are presently over one hundred varieties of hops cultivated around the world. Some of the best known are Brewer’s Gold, Bullion, Cascade, Centennial, Chinook, Cluster, Comet, Eroica, Fuggles, Galena, Goldings, Hallertau, Nugget, Northern Brewer, Perle, Saaz, Syrian Goldings, Tettnang and Willamettes. Apart from contributing bitterness, hops impart aroma and flavor, and inhibit the growth of bacteria in wort and beer. Hops are added at the beginning (bittering hops), middle (flavoring hops), and end (aroma hops) of the boiling stage, or even later in the brewing process (dry hops). The addition of hops to beer dates from 7000-1000 BC; however, hops were used to flavor beer in Pharaonic Egypt around 600 BC. They were cultivated in Germany as early as AD 300 and were used extensively in French and German monasteries in medieval times and gradually superseded other herbs and spices around the fourteenth and fifteenth centuries. Prior to the use of hops, beer was flavored with herbs and spices such as juniper, coriander, cumin, nutmeg, oak leaves, lime blossoms, cloves, rosemary, gentian, gaussia, chamomile, and other herbs or spices.}
  }

\newglossaryentry{Hopping}
{
  name={Hopping},
  description={The addition of hops to un-fermented wort or fermented beer.}
  }

\newglossaryentry{Hot Break}
{
  name={Hot Break},
  description={The flocculation of proteins and tannins during wort boiling.}
  }

\newglossaryentry{Humulene}
{
  name={Humulene},
  description={One of the essential oils made in the flowering cone of the hops plant Humulus lupulus.}
  }

\newglossaryentry{Husk}
{
  name=Husk,
  description={The dry outer layer of certain cereal seeds.}
  }

\newglossaryentry{Hydrometer}
{
  name={Hydrometer},
  description={A glass instrument used to measure the specific gravity of liquids as compared to water, consisting of a graduated stem resting on a weighted float.}
  }

\newglossaryentry{Immersion Chiller}
{
  name={Immersion Chiller},
  description={A wort chiller most commonly made of copper that is used by submerging into hot wort before fermentation as a method of cooling.}
  }

\newglossaryentry{Infusion Mash}
{
  name={Infusion Mash},
  description={A method of mashing which achieves target mashing temperatures by the addition of heated water at specific temperatures.}
  }

\newglossaryentry{Inoculate}
{
  name={Inoculate},
  description={The introduction of a microbe such as yeast or microorganisms such as lactobacillus into surroundings capable of supporting its growth.}
  }

\newglossaryentry{International Bitterness Units (IBU)}
{
  name={International Bitterness Units (IBU)},
  description={The measure of the bittering substances in beer (analytically assessed as milligrams of isomerized alpha acid per liter of beer, in ppm). This measurement depends on the style of beer. Light lagers typically have an IBU rating between 5-10 while big, bitter India Pale Ales can often have an IBU rating between 50 and 70.}
  }

\newglossaryentry{Irish Moss}
{
  name={Irish Moss},
  description={Used as a clairifier in beer. Modified particles or powder of the seaweed Chondrus crispus that help to precipitate proteins in the kettle by facilitating the hot break.}
  }

\newglossaryentry{Isinglass}
{
  name={Isinglass},
  description={A gelatinous substance made from the swim bladder of certain fish that is sometimes added to beer to help clarify and stabilize the finished product.}
  }

\newglossaryentry{Keg}
{
  name={Keg},
  description={A cylindrical container, usually constructed of steel or sometimes aluminum, commonly used to store, transport and serve beer under pressure. In the U.S., kegs are referred to by the portion of a barrel they represent, for example, a ½ barrel keg = 15.5 gal, a ¼ barrel keg = 7.75 gal, a 1/6 barrel keg = 5.23 gal. Other standard keg sizes will be found in other countries.}
  }

\newglossaryentry{Kilning}
{
  name={Kilning},
  description={The process of heat-drying malted barley in a kiln to stop germination and to produce a dry, easily milled malt from which the brittle rootlets are easily removed. Kilning also removes the raw flavor (or green-malt flavor) associated with germinating barley, and new aromas, flavors, and colors develop according to the intensity and duration of the kilning process.}
  }

\newglossaryentry{Kraeusen n}
{
  name={Kraeusen n},
  description={The rocky head of foam which appears on the surface of the wort during fermentation. v – A method of conditioning in which a small quantity of unfermented wort is added to a fully fermented beer to create a secondary fermentation and natural carbonation.}
  }

\newglossaryentry{Lace}
{
  name={Lace},
  description={The lacelike pattern of foam sticking to the sides of a glass of beer once it has been partly or totally emptied. Synonym: Belgian lace}
  }

\newglossaryentry{Lactobacillus}
{
  name={Lactobacillus},
  description={A microorganism/bacteria. Lactobacillus is most often considered to be a beer spoiler, in that it can convert unfermented sugars found in beer into lactic acid. Some brewers introduce Lactobacillus intentionally into finished beer in order to add desirable acidic sourness to the flavor profile of certain brands.}
  }

\newglossaryentry{Lager}
{
  name={Lager},
  description={Lagers are any beer that is fermented with bottom-fermenting yeast at colder temperatures. Lagers are most often associated with crisp, clean flavors and are traditionally fermented and served at colder temperatures than ales.}
  }

\newglossaryentry{Lager Yeast}
{
  name={Lager Yeast},
  description={Saccharomyces pastorianus is a bottom fermenting yeast that ferments in cooler temperatures (45-55 F) and often lends sulfuric compounds.}
  }

\newglossaryentry{Lagering}
{
  name={Lagering},
  description={Storing bottom-fermented beer in cold cellars at near-freezing temperatures for periods of time ranging from a few weeks to years, during which time the yeast cells and proteins settle out and the beer improves in taste.}
  }

\newglossaryentry{Large Brewery}
{
  name={Large Brewery},
  description={As defined by the Brewers Association: A brewery with an annual beer production of over 6,000,000 barrels.}
  }

\newglossaryentry{Lauter Tun}
{
  name={Lauter Tun},
  description={A large vessel fitted with a false slotted bottom (like a colander) and a drain spigot in which the mash is allowed to settle and sweet wort is removed from the grains through a straining process. In some smaller breweries, the mash tun can be used for both mashing and lautering.}
  }

\newglossaryentry{Lautering}
{
  name={Lautering},
  description={The process of separating the sweet wort (pre-boil) from the spent grains in a lauter tun or with other straining apparatus.}
  }

\newglossaryentry{Lightstruck (Skunked)}
{
  name={Lightstruck (Skunked)},
  description={Appears in both the aroma and flavor in beer and is caused by exposure of beer in light colored bottles or beer in a glass to ultra-violet or fluorescent light.}
  }

\newglossaryentry{Liquor}
{
  name={Liquor},
  description={The name given, in the brewing industry, to water used for mashing and brewing, especially natural or treated water containing high amounts of calcium and magnesium salts.}
  }

\newglossaryentry{Lovibond}
{
  name={Lovibond},
  description={A scale used to measure color in grains and sometimes in beer. See also Standard Reference Method.}
  }

\newglossaryentry{Magnum Bottle}
{
  name={Magnum Bottle},
  description={A 1.5L bottle.}
  }

\newglossaryentry{Malt}
{
  name={Malt},
  description={Processed barley that has been steeped in water, germinated on malting floors or in germination boxes or drums, and later dried in kilns for the purpose of stopping the germination and converting the insoluble starch in barley to the soluble substances and sugars in malt.}
  }

\newglossaryentry{Malt Extract}
{
  name={Malt Extract},
  description={A thick syrup or dry powder prepared from malt and sometimes used in brewing (often used by new homebrewers).}
  }

\newglossaryentry{Maltose}
{
  name={Maltose},
  description={The most abundant fermentable sugar in beer.}
  }

\newglossaryentry{Mash}
{
  name={Mash},
  description={A mixture of ground malt (and possibly other grains or adjuncts) and hot water that forms the sweet wort after straining.}
  }

\newglossaryentry{Mash Tun}
{
  name={Mash Tun},
  description={The vessel in which grist is soaked in water and heated in order to convert the starch to sugar and to extract the sugars, colors, flavors and other solubles from the grist.}
  }

\newglossaryentry{Mashing}
{
  name={Mashing},
  description={The process of mixing crushed malt (and possibly other grains or adjuncts) with hot water to convert grain starches to fermentable sugars and non-fermentable carbohydrates that will add body, head retention and other characteristics to the beer. Mashing also extracts colors and flavors that will carry through to the finished beer, and also provides for the degradation of haze-forming proteins. Mashing requires several hours and produces a sugar-rich liquid called wort.}
  }

\newglossaryentry{Mashing Out}
{
  name={Mashing Out},
  description={The process of raising the mash temperature to 170F. The goal is to halt any enzymatic activity and prevent further conversion of starches to sugars.}
  }

\newglossaryentry{Master Brewers Association of the Americas (MBAA)}
{
  name={Master Brewers Association of the Americas (MBAA)},
  description={Master Brewers Association of the Americas (MBAA) was formed in 1887 with the purpose of promoting, advancing, and improving the professional interest of brew and malt house production and technical personnel.}
  }

\newglossaryentry{Microbrewery}
{
  name={Microbrewery},
  description={As defined by the Brewers Association: A brewery that produces less than 15,000 barrels of beer per year with 75 percent or more of its beer sold off-site.}
  }

\newglossaryentry{Milling}
{
  name={Milling},
  description={The grinding of malt into grist (or meal) to facilitate the extraction of sugars and other soluble substances during the mash process. The endosperm must be crushed to medium-sized grits rather than to flour consistency. It is important that the husks remain intact when the grain is milled or cracked because they will later act as a filter aid during lautering.}
  }

\newglossaryentry{Modification}
{
  name={Modification},
  description={ The physical and chemical changes in barley that result from malting, especially the development of enzymes that are required to modify the grain’s starches into sugars during mashing, and also the physical changes that render the carbohydrate found in barley kernels more available to the brewing process.  The degrees to which these changes have occurred, as determined by the growth of the acrospire.}
  }

\newglossaryentry{Modified Malts}
{
  name={Modified Malts},
  description={Modified Malts refers to the length of the germination process and how many of the internal malt structures and compounds have already been broken down.}
  }

\newglossaryentry{Mouthfeel}
{
  name={Mouthfeel},
  description={The textures one perceives in a beer. Includes carbonation, fullness and aftertaste.}
  }

\newglossaryentry{Musty}
{
  name={Musty},
  description={Moldy, mildewy character that can be the result of cork or bacterial infection in a beer. It can be perceived in both taste and aroma.}
  }

\newglossaryentry{Myrcene}
{
  name={Myrcene},
  description={One of the essential oils made in the flowering cone of the hops plant Humulus lupulus.}
  }

\newglossaryentry{Natural Carbonation}
{
  name={Natural Carbonation},
  description={Sugar is added to beer in its container and then sealed. Fermentation kicks off again as the yeast eats the new sugar addition. When yeast ferments, it releases CO2 which is then absorbed into the liquid.}
  }

\newglossaryentry{Ninkasi}
{
  name={Ninkasi},
  description={The ancient Sumerian goddess of beer.}
  }

\newglossaryentry{Nitrogen}
{
  name={Nitrogen},
  description={When used for the carbonation of beer, Nitrogen contributes a thick creamy mouthfeel, different from the mouthfeel you get from CO2.}
  }

\newglossaryentry{Noble Hops}
{
  name={Noble Hops},
  description={Traditional European hop varieties prized for their characteristic flavor and aroma. Traditionally these are grown only in four small areas in Europe: Hallertau in Bavaria, Germany Saaz in Zatec, Czech Republic Spalt in Spalter, Germany Tettnang in the Lake Constance region, Germany}
  }

\newglossaryentry{Oasthouse}
{
  name={Oasthouse},
  description={A farm-based facility where hops are dried and baled after picking.}
  }

\newglossaryentry{Original Gravity (OG)}
{
  name={Original Gravity (OG)},
  description={The specific gravity of wort before fermentation. A measure of the total amount of solids that are dissolved in the wort as compared to the density of water, which is conventionally given as 1.000 and higher. Synonym: Starting gravity; starting specific gravity; original wort gravity.}
  }

\newglossaryentry{Oxidation}
{
  name={Oxidation},
  description={A chemical reaction in which one of the reactants (beer, food) undergoes the addition of or reaction with oxygen or an oxidizing agent.}
  }

\newglossaryentry{Oxidized}
{
  name={Oxidized},
  description={Stale, winy flavor or aroma of wet cardboard, paper, rotten pineapple sherry and many other variations.}
  }

\newglossaryentry{Package}
{
  name={Package},
  description={A general term for the containers used to market beverages. Packaged beer is generally sold in bottles and cans. Beer sold in kegs is usually called draught beer.}
  }

\newglossaryentry{Palate}
{
  name={Palate},
  description={The top part of the inside of your mouth and is generally associated with how humans taste.}
  }

\newglossaryentry{Pediococcus}
{
  name={Pediococcus},
  description={A microorganism orbacteria usually considered contaminants of beer and wine although their presence is sometimes desired in beer styles such as Lambic. Certain Pediococcus strains can produce diacetyl, which renders a buttery or butterscotch aroma and flavor to beer, sometimes desired in small doses, but usually considered to be a flavor defect.}
  }

\newglossaryentry{pH}
{
  name={pH},
  description={Abbreviation for potential Hydrogen, used to express the degree of acidity and alkalinity in an aqueous solution, usually on a logarithmic scale ranging from 1-14, with 7 being neutral, 1 being the most acidic, and 14 being the most alkaline.}
  }

\newglossaryentry{Phenols}
{
  name={Phenols},
  description={A class of chemical compounds perceptible in both aroma and taste. Some phenolic flavors and aromas are desirable in certain beer styles, for example German-style wheat beers in which the phenolic components derived from the yeast used, or Smoke beers in which the phenolic components derived from smoked malt. Higher concentrations in beer are often due to the brewing water, infection of the wort by bacteria or wild yeasts, cleaning agents, or crown and can linings. Phenolic sensory attributes include clovey, herbal, medicinal or pharmaceutical (band-aid).}
  }

\newglossaryentry{Pitching}
{
  name={Pitching},
  description={The addition of yeast to the wort once it has cooled down to desirable temperatures.}
  }

\newglossaryentry{Primary Fermentation}
{
  name={Primary Firmentation},
  description={The first stage of fermentation carried out in open or closed containers and lasting from two to twenty days during which time the bulk of the fermentable sugars are converted to ethyl alcohol and carbon dioxide gas. Synonym: Principal fermentation; initial fermentation.}
  }

\newglossaryentry{Priming}
{
  name={Priming},
  description={The addition of small amounts of fermentable sugars to fermented beer before racking or bottling to induce a renewed fermentation in the bottle or keg and thus carbonate the beer.}
  }

\newglossaryentry{Prohibition}
{
  name={Prohibition},
  description={A law instituted by the Eighteenth Amendment to the U.S. Constitution (stemming from the Volstead Act) on January 18, 1920, forbidding the sale, production, importation, and transportation of alcoholic beverages in the U.S. It was repealed by the Twenty-first Amendment to the U.S. Constitution on December 5, 1933. The Prohibition Era is sometimes referred to as ``The Noble Experiment.''}
  }

\newglossaryentry{Punt}
{
  name={Punt},
  description={The hollow at the bottom of some bottles.}
  }

\newglossaryentry{Quaff}
{
  name={Quaff},
  description={To drink deeply.}
  }

\newglossaryentry{Racking}
{
  name={Racking},
  description={The process of transferring beer from one vessel to another, especially into the final package or keg.}
  }

\newglossaryentry{Real Ale}
{
  name={Real Ale},
  description={A style of beer found primarily in England, where it has been championed by the consumer rights group called the Campaign for Real Ale (CAMRA). Generally defined as beers that have undergone a secondary fermentation in the container from which they are served and that are served without the application of carbon dioxide.}
  }

\newglossaryentry{Regional Craft Brewery}
{
  name={Regional Craft Brewery},
  description={As defined by the Brewers Association: An independent regional brewery having either an all malt flagship or has at least 50 percent of its volume in either all malt beers or in beers which use adjuncts to enhance rather than lighten flavor.}
  }

\newglossaryentry{Reinheitsgebot}
{
  name={Reinheitsgebot},
  description={The German beer purity law passed in 1516, stating that beer may only contain water, barley and hops. Yeast was later added after its role in fermentation was discovered by Louis Pasteur.}
  }

\newglossaryentry{Residual Alkalinity}
{
  name={Residual Alkalinity},
  description={A measurement of the mash’s ability to buffer, or resist, attempts to lower its pH.}
  }

\newglossaryentry{Residual Sugar}
{
  name={Residual Sugar},
  description={Any leftover sugar that the yeast did not consume during fermentation.}
  }

\newglossaryentry{Resin}
{
  name={Resin},
  description={The gummy organic substance produced by certain plants and trees. Humulone and lupulone, for example, are bitter resins that occur naturally in the hop flower.}
  }

\newglossaryentry{Saccharification}
{
  name={Saccharification},
  description={The conversion of malt starch into fermentable sugars, primarily maltose.}
  }

\newglossaryentry{Saccharomyces}
{
  name={Saccharomyces},
  description={The genus of single-celled yeasts that ferment sugar and are used in the making of alcoholic beverages and bread. Yeasts of the species Saccharomyces cerevisiae and Saccharomyces pastorianus are commonly used in brewing.}
  }

\newglossaryentry{Secondary Fermentation}
{
  name={Secondary Firmentation},
  description={ The second, slower stage of fermentation for top fermenting beers, and lasting from a few weeks to many months, depending on the type of beer.  A renewed fermentation in bottles or casks and initiated by priming or by adding fresh yeast.  }
  }

\newglossaryentry{Sediment}
{
  name={Sediment},
  description={The refuse of solid matter that settles and accumulates at the bottom of fermenters, conditioning vessels and bottles of bottle-conditioned beer.}
  }

\newglossaryentry{Session Beer}
{
  name={Session Beer},
  description={A beer of lighter body and alcohol of which one might expect to drink more than one serving in a sitting.}
  }

\newglossaryentry{Solvent-like}
{
  name={Solvent-like},
  description={Flavor and aromatic character similar to acetone or lacquer thinner, often due to high fermentation temperatures.}
  }

\newglossaryentry{Sorghum}
{
  name={Sorghum},
  description={A cereal grain from various grasses (Sorghum vulgare). Also a grain sought out by those who are gluten intolerant.}
  }

\newglossaryentry{Sour}
{
  name={Sour},
  description={A taste perceived to be acidic and tart. Sometimes the result of a bacterial influence intended by the brewer, from either wild or inoculated bacteria such as lactobacillus and pediococcus.}
  }

\newglossaryentry{sparge water}
{
  name={sparge water},
  description={The hot water that is run through the grain bed to extract a sweet liquid called wort.}
  }

\newglossaryentry{Sparging}
{
  name={Sparging},
  description={In lautering, an operation consisting of spraying the spent mash grains with hot water to retrieve the liquid malt sugar and extract remaining in the grain husks.}
  }

\newglossaryentry{Specific Gravity}
{
  name={Specific Gravity},
  description={The ratio of the density of a substance to the density of water. This method is used to determine how much dissolved sugars are present in the wort or beer. Specific gravity has no units because it is expressed as a ratio. See also Original Gravity and Final Gravity.}
  }

\newglossaryentry{Standard Reference Method (SRM)}
{
  name={Standard Reference Method (SRM)},
  description={An analytical method and scale that brewers use to measure and quantify the color of a beer. The higher the SRM is, the darker the beer. In beer, SRM ranges from as low as 2 (light lager) to as high as 45 (stout) and beyond.}
  }

\newglossaryentry{Steeping}
{
  name={Steeping},
  description={The soaking in liquid of a solid so as to extract flavors.}
  }

\newglossaryentry{Step Infusion}
{
  name={Step Infusion},
  description={A mashing method wherein the temperature of the mash is raised by adding very hot water, and then stirring and stabilizing the mash at the target step temperature.}
  }

\newglossaryentry{Sulfur}
{
  name={Sulfer},
  description={Aroma reminiscent of rotten eggs or burnt matches; a by-product of some yeasts or a beer becoming light struck.}
  }

\newglossaryentry{Tannins}
{
  name={Tannins},
  description={A group of organic compounds contained in certain cereal grains and other plants. Tannins are present in the hop cone. Also called “hop tannin” to distinguish it from tannins originating from malted barley. The greater part of malt tannin content is derived from malt husks, but malt tannins differ chemically from hop tannins. In extreme examples, tannins from both can be perceived as a taste or sensation similar to sampling black tea that has steeped for a very long time.}
  }

\newglossaryentry{Temperature Rests}
{
  name={Temperature Rests},
  description={Temperature Rests during the beer making process allows the brewer to adjust fermentable sugar profiles so as to influence characteristics of the resulting beer.}
  }

\newglossaryentry{Top Fermentation}
{
  name={Top Fermentation},
  description={One of the two basic fermentation methods characterized by the tendency of yeast cells to rise to the surface of the fermentation vessel. Ale yeast is top fermenting compared to lager yeast, which is bottom fermenting. Beers brewed in this fashion are commonly called ale or top-fermented beers.}
  }

\newglossaryentry{Trigeminal Nerves}
{
  name={Trigeminal Nerves},
  description={These nerves of the human face sense temperature and texture. Detection descriptors tied to beer’s sensations include: Cold/Hot, Silky/Tannic/Astringent, Thin/Heavy, Dry/Cloying, Flabby/Puckering, Cool/Burn}
  }

\newglossaryentry{Trub}
{
  name={Trub},
  description={Wort particles resulting from the precipitation of proteins, hop oils and tannins during the boiling and cooling stages of brewing.}
  }

\newglossaryentry{Turbidity}
{
  name=Turbidity,
  description={Sediment in suspension; hazy, murky.}
  }

\newglossaryentry{Volatile Compounds}
{
  name={Volatile Compounds},
  description={Chemicals that have a high vapor pressure at ordinary room temperature which causes large numbers of molecules to evaporate and enter the surrounding air.}
  }

\newglossaryentry{Volstead Act}
{
  name={Volstead Act},
  description={Or the national prohibition act, was enacted to carry out the intent of the Eighteenth Amendment, which established prohibition in the United States.}
  }

\newglossaryentry{Volumes of C02}
{
  name={Volumes of CO2},
  description={The measurement of c02 dissolved in a beer and is an indication of the carbonation level.}
  }

\newglossaryentry{Vorlauf}
{
  name={Vorlauf},
  description={At the outset of lautering and immediately prior to collecting wort in the brew kettle, the recirculation of wort from the lauter tun outlet back onto the top of the grain bed in order to clarify the wort.}
  }

\newglossaryentry{Water}
{
  name={Water},
  description={One of the four ingredients in beer. Some beers are made up by as much as 90\% water. Globally, some brewing centers became famous for their particular type of beer, and the individual flavors of their beer were strongly influenced by the brewing water’s pH and mineral content. Burton is renowned for its bitter beers because the water is hard (higher PH), Edinburgh for its pale ales, Dortmund for its pale lager, and Plzen for its Pilsner Urquell (soft water lower PH).}
  }

\newglossaryentry{Wet Hopping}
{
  name={Wet Hopping},
  description={The addition of freshly harvested hops that have not yet been dried to different stages of the brewing process. Wet hopping adds unique flavors and aromas to beer that are not normally found when using hops that have been dried and processed per usual.}
  }

\newglossaryentry{Whirlpool}
{
  name={Whirlpool},
  description={A method of collecting hot break material in the center of the kettle by stirring the wort until a vortex is formed.  A brewhouse vessel designed to separate hot break trub particles from boiled wort.}
  }

\newglossaryentry{Wort}
{
  name={Wort},
  description={The bittersweet sugar solution obtained by mashing the malt and boiling in the hops, which becomes beer through fermentation.}
  }

\newglossaryentry{Yeast}
{
  name={Yeast},
  description={During the fermentation process, yeast converts the natural malt sugars into alcohol and carbon dioxide gas. Yeast was first viewed under a microscope in 1680 by the Dutch scientist Antonie van Leeuwenhoek; in 1867, Louis Pasteur discovered that yeast cells lack chlorophyll and that they could develop only in an environment containing both nitrogen and carbon.}
  }

\newglossaryentry{Yeast Cake}
{
  name={Yeast Cake},
  description={Living yeast cells compressed with starch into a cake, for use in brewing.}
  }

\newglossaryentry{Yeast Pitching}
{
  name={Yeast Pitching},
  description={The point in the brewing process in which yeast is added to cool wort prior to fermentation.}
  }

\newglossaryentry{Zymurgy}
{
  name={Zymurgy},
  description={The branch of chemistry that deals with fermentation processes, as in brewing. Also the name of the American Homebrewers Association bi-monthly magazine.}
  }

\makeglossaries
