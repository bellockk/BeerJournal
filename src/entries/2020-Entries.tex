%------------------------------------------------------------------------------
\def\todaysdate{20200106}
\newday{\todaysdate}\label{\todaysdate}

\newentry[Racking]{2005 Racking Cider}

\begin{my_itemize}
    \item Racked the cider.  As it was 5 gallon jugs, plus 8 lbs of brown sugar, there was more than could fit in a 5 gallon corny keg.  Two notes made during the process were that when I know we are so close to 5 gallons stop the racking when there is plenty left in the fermentation bucket.  One at the end there was a lot of yeast that got sucked up into the racking cane, and two a lot of cider spouted out of the vent end of the keg making a big sticky puddle on the floor.  Next time the keg should go into a 5 gallon bucket so overflow is caught in the bucket instead of making a puddle on the floor.
\end{my_itemize}

%------------------------------------------------------------------------------
\def\todaysdate{20200111}
\newday{\todaysdate}\label{\todaysdate}

\newentry[Racking]{0930 Racking Tap 6 Clone}

\begin{my_itemize}
    \item Produced maybe 3 gallons of product.  This should be a 5 gallon recipe.  The color is much darker this batch.
\end{my_itemize}

%------------------------------------------------------------------------------
\def\todaysdate{20200209}
\newday{\todaysdate}\label{\todaysdate}

\newentry[Brew]{0500 Brewing One Eyed Brown Girl}

0500 Start 6 Gallon in mash tun at 160, connected hoses and fittings
0640 Reached Temp, connected recirculation line to drain and pumped hot liquor to lower level in mash tun to 4.5 gallons, flush debris from pump and plate chiller and prime the lines.  Ground grain.
0654 Set mash tun to 154 deg, started Mash In
0700 Mash in complete temp at 153, sanitized fermentation bucket, yeast pack, air lock.  Prepped fermentation chamber, readied tilt sensor.
0800 Raised temperature setting to 198, Raised mash tube, Closed valve on the mash tun and opened the valve on the sparge water heater.  Targeting 11L of sparge water.
0845 11L of sparge water transferred, turned off pump and closed the valve on the sparge water.  Allowing remainder of hour to drain malt pipe.
0900 Set the mash tun temperature to 220.
0915 Removed malt pipe at 205 degrees, placed empty hop spider into kettle to break surface tension and reduce risk of boil over.  Collected 6.9 gallons of wort according to mash tun markings.  This is more than I expected.  I need to back off on the sparge water amount on the next brew.
0915 Boil Started. SG 1.034 Added challenger hops to hop spider.
1000 SG is only down to 1.039, adding an hour to the boil time to account for sparging with too much water.
1100 SG is only up to 1.041 adding another hour to the boil
1225 Added Irish moss (With 15 Minutes left in Boil)
1235 Added Willamette hops (With 5 Minutes left in Boil)
11240 Boil ended. OG 1.044, well below the target of 1.060.
1005 Transfer and oxygenate to primary fermentation and pitched yeast.  Fermentation chamber targeting 68 degrees.
0710 Finished Cleaning
1649 Post brew thoughts:  Efficiency is way down from the last brews.  This has been an ongoing problem.  At the next brew I will narrow the gap in the mill and grind to a finer level.  I was in no risk during todays brew of having a stuck mash, so it should be safe to try a finer crush to increase the efficiency.  An 11lb grain bill seems like an easy walk after the last two 22lb grain bills.  This was a very easy brew, aside from the excess water used during the sparge.  On the excess water.  I sparged with 11 liters and collected just under 7 gallons according to the mash tun markings.  That means I actually collected almost 7.5 gallons post sparge.  For the next brew of the recipe, I will start with 4 gallons in the mash tun and sparge with 7 liters of water.  During the 3 hours of boiling today, my evaporation rate was about 1 gallon an hour.

%------------------------------------------------------------------------------
\def\todaysdate{20200212}
\newday{\todaysdate}\label{\todaysdate}

\newentry[Brew]{1900 Cider}
\begin{itemize}
    \item Ingredients: 6 96oz containers of apple juice.  4 sticks of cinnamon, 3.5lbs of brown sugar.
    \item Added brown sugar and cinnamon to the pot, and added just enough juice to cover the brown sugar.
    \item Heated on high until all the brown sugar had melted and the mixture had boiled for 5 minutes.
    \item Pitched 50g of L2 Lallemond Enoferm yeast, and stirred.
    \item Aerated with the oxygen wand for 5 minutes.
    \item Placed into fermentation chamber set to 68 degrees.
\end{itemize}

%------------------------------------------------------------------------------
\def\todaysdate{20200330}
\newday{\todaysdate}\label{\todaysdate}

\newentry[Brew]{1900 Cider}
\begin{itemize}
    \item Ingredients: 6 96oz containers of apple juice. 3.5lbs of brown sugar.
    \item Added brown sugar and cinnamon to the pot, and added just enough juice to cover the brown sugar.
    \item Heated on high until all the brown sugar had melted and the mixture had boiled for 5 minutes.
    \item Pitched 50g of L2 Lallemond Enoferm yeast, and stirred.
    \item Aerated with the oxygen wand for 5 minutes.
    \item Placed into fermentation chamber set to 68 degrees.
\end{itemize}

