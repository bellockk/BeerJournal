%------------------------------------------------------------------------------
\def\todaysdate{20200106}
\newday{\todaysdate}\label{\todaysdate}

\newentry[Racking]{2005 Racking Cider}

\begin{my_itemize}
    \item Racked the cider.  As it was 5 gallon jugs, plus 8 lbs of brown sugar, there was more than could fit in a 5 gallon corny keg.  Two notes made during the process were that when I know we are so close to 5 gallons stop the racking when there is plenty left in the fermentation bucket.  One at the end there was a lot of yeast that got sucked up into the racking cane, and two a lot of cider spouted out of the vent end of the keg making a big sticky puddle on the floor.  Next time the keg should go into a 5 gallon bucket so overflow is caught in the bucket instead of making a puddle on the floor.
\end{my_itemize}

%------------------------------------------------------------------------------
\def\todaysdate{20200111}
\newday{\todaysdate}\label{\todaysdate}

\newentry[Racking]{0930 Racking Tap 6 Clone}

\begin{my_itemize}
    \item Produced maybe 3 gallons of product.  This should be a 5 gallon recipe.  The color is much darker this batch.
\end{my_itemize}

%------------------------------------------------------------------------------
\def\todaysdate{20200209}
\newday{\todaysdate}\label{\todaysdate}

\newentry[Brew]{0500 Brewing One Eyed Brown Girl}

0500 Start 6 Gallon in mash tun at 160, connected hoses and fittings
0640 Reached Temp, connected recirculation line to drain and pumped hot liquor to lower level in mash tun to 4.5 gallons, flush debris from pump and plate chiller and prime the lines.  Ground grain.
0654 Set mash tun to 154 deg, started Mash In
0700 Mash in complete temp at 153, sanitized fermentation bucket, yeast pack, air lock.  Prepped fermentation chamber, readied tilt sensor.
0800 Raised temperature setting to 198, Raised mash tube, Closed valve on the mash tun and opened the valve on the sparge water heater.  Targeting 11L of sparge water.
0845 11L of sparge water transferred, turned off pump and closed the valve on the sparge water.  Allowing remainder of hour to drain malt pipe.
0900 Set the mash tun temperature to 220.
0915 Removed malt pipe at 205 degrees, placed empty hop spider into kettle to break surface tension and reduce risk of boil over.  Collected 6.9 gallons of wort according to mash tun markings.  This is more than I expected.  I need to back off on the sparge water amount on the next brew.
0915 Boil Started. SG 1.034 Added challenger hops to hop spider.
1000 SG is only down to 1.039, adding an hour to the boil time to account for sparging with too much water.
1100 SG is only up to 1.041 adding another hour to the boil
1225 Added Irish moss (With 15 Minutes left in Boil)
1235 Added Willamette hops (With 5 Minutes left in Boil)
11240 Boil ended. OG 1.044, well below the target of 1.060.
1005 Transfer and oxygenate to primary fermentation and pitched yeast.  Fermentation chamber targeting 68 degrees.
0710 Finished Cleaning
1649 Post brew thoughts:  Efficiency is way down from the last brews.  This has been an ongoing problem.  At the next brew I will narrow the gap in the mill and grind to a finer level.  I was in no risk during todays brew of having a stuck mash, so it should be safe to try a finer crush to increase the efficiency.  An 11lb grain bill seems like an easy walk after the last two 22lb grain bills.  This was a very easy brew, aside from the excess water used during the sparge.  On the excess water.  I sparged with 11 liters and collected just under 7 gallons according to the mash tun markings.  That means I actually collected almost 7.5 gallons post sparge.  For the next brew of the recipe, I will start with 4 gallons in the mash tun and sparge with 7 liters of water.  During the 3 hours of boiling today, my evaporation rate was about 1 gallon an hour.

%------------------------------------------------------------------------------
\def\todaysdate{20200212}
\newday{\todaysdate}\label{\todaysdate}

\newentry[Brew]{1900 Cider}
\begin{itemize}
    \item Ingredients: 6 96oz containers of apple juice.  4 sticks of cinnamon, 3.5lbs of brown sugar.
    \item Added brown sugar and cinnamon to the pot, and added just enough juice to cover the brown sugar.
    \item Heated on high until all the brown sugar had melted and the mixture had boiled for 5 minutes.
    \item Pitched 50g of L2 Lallemond Enoferm yeast, and stirred.
    \item Aerated with the oxygen wand for 5 minutes.
    \item Placed into fermentation chamber set to 68 degrees.
\end{itemize}

%------------------------------------------------------------------------------
\def\todaysdate{20200228}
\newday{\todaysdate}\label{\todaysdate}

\newentry[Racking]{1400 Racking}
\begin{itemize}
    \item Racked the cider and porter.  The power had gone out during fermentation and it stopped the data logging.  I took the current readings at racking time and updated the spreadsheets with a data point with todays readings.  The power outage also corrupted the raspberry pie installation of the tilt operating system so it was reinstalled and upgraded.  The new version is supposed to have a fix for restoring logging activities after a power outage.
\end{itemize}

%------------------------------------------------------------------------------
\def\todaysdate{20200330}
\newday{\todaysdate}\label{\todaysdate}

\newentry[Brew]{1900 Cider}
\begin{itemize}
    \item Ingredients: 6 96oz containers of apple juice. 3.5lbs of brown sugar.
    \item Added brown sugar to the pot, and added just enough juice to cover the brown sugar.
    \item Heated on high until all the brown sugar had melted and the mixture had boiled for 5 minutes.
    \item Pitched 50g of L2 Lallemond Enoferm yeast, and stirred.
    \item Aerated with the oxygen wand for 5 minutes.
    \item Placed into fermentation chamber set to 68 degrees.
    \item Used a mixture of everclear and water to fill the air lock.
\end{itemize}

%------------------------------------------------------------------------------
\def\todaysdate{20200331}
\newday{\todaysdate}\label{\todaysdate}

\newentry[Brew]{1930 Cider}
\begin{itemize}
    \item Ingredients: 6 96oz containers of apple juice. 7.0lbs of brown sugar.
    \item Added brown sugar to the pot, and added just enough juice to cover the brown sugar.
    \item Heated on high until all the brown sugar had melted and the mixture had boiled for 5 minutes.
    \item Pitched 50g of L2 Lallemond Enoferm yeast, and stirred.
    \item Aerated with the oxygen wand for 5 minutes.
    \item Placed into fermentation chamber set to 68 degrees.
    \item Used a mixture of everclear and water to fill the air lock.
\end{itemize}

%------------------------------------------------------------------------------
\def\todaysdate{20200430}
\newday{\todaysdate}\label{\todaysdate}

\newentry[Brew]{0830 Brew Day}

The two batches of cider were checked this morning.  Both airlocks were extremely low, I will stop using the mixture of everclear and water and return to using One Step to fill the airlocks.  The 3/31 batch is still bubbling away, so I will let it continue to ferment.  The 3/30 batch is idle and I will rack it today.  I will need this fermentation chamber for today's brew.

\begin{itemize}
    \item Mixed up a batch of sanitizer with 1 gallon of hot tap water and 1 tbsp of One Step.
    \item Assembled the overflow bypass.  This is a new addition to the brewday.
    \item Assembled brew hoses and fixtures. 
\end{itemize}

\begin{description}
    \item[5000g] Wheat Malt (4)
    \item[2000g] Munich Malt (1)
    \item[2000g] Pilsner Malt (2)
    \item[1000g] Caramunich Malt (3)
    \item[250g] Rice Hulls
\end{description}

0900 Start Heating water.  Filled to 5.0 gallon mark on mash tun, which equates to 5.5 gallons of water, set temperature to 130F.  Put 16.5 liters in the sparge water heater, and started heat at 80C.  To make sure the lines did not take away from the volume, I circulated the pump to fill the lines.  When I did this a lot of black film like substance came out of the plate chiller.  In the future I will need to rince out the plate chiller before recirculating, and the idea also came up to switch sides on the plate chiller for running wort through such that every other brew the plate chiller will be thoroughly rinsed out with clean water from the well water side during cooling.

1000 Water is up to temp.  Re-gapped mill to .035 using a feeler gauge, and milled the grains.  Added one scoop of 
1040 Started Mash in
1050 Mash in Complete inserted the train bar. Temp is 120
1124 Raised temperate in mash tube to 150.  Started recirculating. The overflow bypass filled very quickly, but it does seem to be holding the water at a level flush with it's top.  Using it seems to help, but it is not a full proof prevention of overflow, it just allows a bit more buffer space on the flow setting of the recirculation.  Recirculation still has to be very slow.
1135 We have reached 150, restarting one hour timer from here.
1235 Raised mash tube up. Starged sparging, targeting 8.0L of water for sparging over approximately an hour.  Setting temperature of kettle to 185.

1320 8.0 Liters completed sparging, letting wort finish draining from grain bed for remaining hour.

1335 Set the kettle temperature to 220 to bring the wort to a boil.

1355 Removed grain bed at 214 degrees, boil started, wort level is at the 5.0 gal mark on the mash tun. 1.034

1410 Added 1.5 oz German Hersbrucker Hops 1.8\% in hop spider.  Started Timer for 55 minutes.

1505 Added 1 teaspoon irish moss and .5 oz  German Hersbrucker Hops 1.8\% in hop spider

1515 Started circulating through the plate chiller to sanitize everything for remaining 5 minutes.  Heat is on full.

1520 Boil Complete.  Trasfer to primary fermentation bucket with yeast at 64 degrees.

1744 Finished Cleaning.

Target OG: 1.059
Target Temperature: 64

Actual OG: 1.037
Actual Temperature: 64

