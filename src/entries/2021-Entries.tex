%------------------------------------------------------------------------------
\def\todaysdate{20210108}
\newday{\todaysdate}\label{\todaysdate}

\newentry[Racking]{1430 Racking Cider}

\begin{my_itemize}
    \item Racked both ciders from 11/22 and 12/07.  Placed the 11/22 into the kegerator.
\end{my_itemize}

%------------------------------------------------------------------------------
\def\todaysdate{20210530}
\newday{\todaysdate}\label{\todaysdate}

\newentry[Cranberry Apple Cider Brewing]{1130 Brewing Cranberry Apple Cider}

\begin{my_itemize}
    \item Mixed 32oz of apple juice and 7lbs of brown sugar to a boil and held there for 5 minutes.  Added remaining 96oz * 5 of apple juice, and 96oz of cranberry juice.  Oxygenated with a wand for 5 minutes.  Pitched M2 Enoferm and Blue Tilt sensor.
    \item The tilt sensor is not showing up.  I was using a Raspberry Pi Zero, which was having trouble connecting to the tilt before so I switched to a Raspberry Pi 3.  Still not making a connection to the tilt sensor.  I will continue to monitor, but if the battery is dead, I will watch the airlock and let it ferment for two months before racking as I do not want to recover it and risk an infection, and I cannot pitch in my other sensor as I will be brewing beer soon and will need it for that.
\end{my_itemize}

%------------------------------------------------------------------------------
\def\todaysdate{20210603}
\newday{\todaysdate}\label{\todaysdate}

\newentry[Tilt Sensor Batter Change]{2000 Tilt Sensor Batter Change}

The Blue tilt sensor is not reading which is likely due to a dead battery.  I sanitized a set of thongs, and opened the fermenter and removed the sensor.  Replaced the batteries in both the blue and black tilt.  It was very difficult to get the lid off the Blue sensor.  It was accomplisehed by wrapping the lid in several layers of paper towels and using a pipe wrench.  The black tilt sensor lid came off with little effort.  Both were calibrated SG in water.  

%------------------------------------------------------------------------------
\def\todaysdate{20210717}
\newday{\todaysdate}\label{\todaysdate}

\newentry[Brew]{Brewing One Eyed Brown Girl}

\begin{description}
    \item[0700] Cleaning and organizing work area.  Connected hoses and fittings, with the wort flowing through the plate chiller on the Q1/Q2 side.  The T goes into the output port of the pump.  The shortest hose goes from the mash tun output port to the left side of the T.  The second shortest hose goes from the right side of the T to the sparge water tank output port.  The next smallest hose goes from the output port of the pump to the input port of the plate chiller.  The output port of the plate chiller is connected to the temperature guage, then the Oxygen infusion T, followed by the ball valve, connected to the longest stretch of hose.
    \item[0755] Added 6 gallons in the mash tun at 160, 16 liters in the sparge water heater at 80C.  Placed malt pipe in a bucket, and covered the bottom of the false bottom with rice hulls.  Placed the grain mill on the top of the malt pipe, and poured the grains in the top.  All grains are already measured out since this grain bill was assembled by a Home Brew store from recipe into one bag.  Mixed two heaping handfulls of rice hulls into the grain in the hopper and mixed roughly by hand.  Milled using an electric drill to drive the shaft.  No modifications to the mill settings were made, so gap is set at .035.  Cleaned the mill using compressed air.
    \item[0850] Reached Temp, connected recirculation line to drain and pumped hot liquor to lower level in mash tun to the 4.0 gallon mark (4.5 gallon actual), flush debris from pump and plate chiller and prime the lines.
    \item[0857] Set mash tun to 154 deg, started Mash in. No stirring was done at mash in.  The grain bed is undisturbed from grinding, and is slowly lowered down into the hot liquer and then the recirulation pump was turned on.  Recirculation speed was adjusted so that the fastest flow is achieved without flooding the malt pipe.
    \item[0900] Mash in complete temp at 153
    \item[0905] Racked cranberry apple cider to keg.  Took 50 minutes including cleaning.  Sprayed a corny kegs connectors with sanitizer.  Connected the output hose of the transfer pump to the out port on the corny keg.  Attached a connector to the in port of the corny keg, to which a short stretch of beer line was connected and dropped into an overflow bucket.  Pumped sanitizer through the lines and into the keg.  Let sanitizer sit for 2 minutes.  Pumped air through the lines.  Disconnected all lines from the corny keg and shook for a minute.  Opened the top and poured out all sanitizer into the bucket.  Tipped the corny keg back up and dumped again to ensure everything was drained out of the drop pipe.  Sprayed the lid and inside of keg with one shot of sanitizer, and put the lid back on.  Removed lid from fermentation bucket, turned on the pump, and inserted the racking cane into the bucket.  Transferred by this method until nearing the bottom, ensuring not to let the racking cane touch the bottom, then tilted the bucket on its side and sucked out a bit more, leaving enough to ensure none of the yeast cake was sucked into the keg.  Disconnected all lines from the corny keg, and connected a pass through on the output end of the transfer pump, and pumped cleaner through.  Let it sit for 2 minutes in the lines, then pumped a small bit more through, then pumped air through to empty the lines.  Washed the connectors and overflow connector.  Washed the overflow bucket, fermentation bucket, and all connectors.  Took the new keg to the kegerator, and attached the CO2 line and purged the tank several times to remove any O2 and fill the head space with CO2.  Connected the out line and fed enough through to prime the lines with cider.
    \item[0950] Sanitized fermentation bucket, yeast pack, air lock.  Prepped fermentation chamber, readied tilt sensor.
    \item[1000] Stopped pump.  Raised temperature setting to 198, Raised mash tube, Closed valve on the mash tun and opened the valve on the sparge water heater.  Turned pump back on and set the ball valve to transfer at a slow rate.  The goal is to take 30 minutes to transfer 7 of the 8 liters of sparge water.
    \item[1016] 7 liters transferred, opened valve all the way to flood the malt pipe with the last liter of sparge water.  Closed the sparge water valve and opened the mash tun valve and ran the pump to sparge with the remaining clear water in the lines.  Ran the pump till dark wort ran up the tube to the mash tun and then shut the pump off.  Waiting for water to drain into the mash tun.  My goal is to have wort fill to the 5.5 gallon mark on the mash tun since our boil is 1 hour, and we get an evaporation rate of 1 gallon an hour, and we want to have 5 gallons when we are done.  We started with 4.5 gallons plus lines, and 8 liters is a little more than 2 gallons.  I figure an amount equal to the lines plus a half gallon will be left in the grain bed when draining is completed.
    \item[1045] Set the mash tun to 220F.
    \item[1050] Removed malt pipe, placed empty hop spider into kettle to break surface tension and reduce risk of boil over.  Collected 5.75 gallons (actual) of wort.
    \item[1100] Boil Started. SG 1.038 Added challenger hops to hop spider.
    \item[1145] Added Irish moss (With 15 Minutes left in Boil)
    \item[1155] Added Willamette hops (With 5 Minutes left in Boil).  Turned on recirculation pump to sanitize the plate chiller.
    \item[1200] Boil ended. Final volume 5 gallons actual. OG 1.041, well below the target of 1.060.
    \item[1205] Transfer and oxygenate to primary fermentation and pitched yeast.  Fermentation chamber targeting 68 degrees.
    \item[1355] Finished Cleaning
\end{description}

%------------------------------------------------------------------------------
\def\todaysdate{20210801}
\newday{\todaysdate}\label{\todaysdate}

\newentry[Racking]{Racking the One Eyed Brown Girl and Cranberry Apple Cider}

Racked both fermentation buckets to kegs and placed them in the kegerator.  The existing cranberry apple cider in the kegerator was moved to a fermentation chamber, and its temperature was lowered to 38 degrees F.  All fermentation chambers were cleaned.

%------------------------------------------------------------------------------
\def\todaysdate{20211015}
\newday{\todaysdate}\label{\todaysdate}

\newentry[Brew]{0930 Brew Day}

\begin{itemize}
    \item Mixed up a batch of sanitizer with 1 gallon of hot tap water and 1 tbsp of One Step.
    \item Assembled brew hoses and fixtures - using left side of plate chiller for wort
    \item Today's Experiments:
        \begin{itemize}
            \item Will be skipping overflow bypass
            \item Will be milling directly into the malt pipe and mix in rice hulls as it fills
            \item Going to mash in by lowering the malt pipe into the hot liquor, forcing it up through the bottom of the malt pipe
        \end{itemize}
\end{itemize}

\begin{description}
    \item[5000g] Wheat Malt (4)
    \item[2000g] Munich Malt (1)
    \item[2000g] Pilsner Malt (2)
    \item[1000g] Caramunich Malt (3)
\end{description}

\begin{description}
    \item[0930] Start Heating water.  Filled to 6 gallon mark (6.5 gallon actual) so that lines can be rinsed and filled once heating is complete. Lowered the level in the mash tun to 5.0 gallon mark (5.5 gallon actual) on mash tun by pumping to the drain.  This rinsed out a bunch of flaky black substance from the plate chiller.  The 5 gallon mark on the mash tun equates to 5.5 gallons of water, set temperature to 130F.  Put 16.0 liters in the sparge water heater, and started heat at 80C.  
    \item[0945] Grinding grain.  I adjusted the mill by tightening up the gap.  There is a line mark on the top of each adjusting dial which when pointing straight up indicates a .40 mm gap. The dial on the drive shaft side was rotated clockwise about 10 degrees, and the dial on the other side was rotated about 10 degrees counter clockwise.
    \item[1005] Water is up to temp.
    \item[1115] Set mash tun to 120. Started Mash in.  I dropped the malt pipe into the mash tun without stirring.  I slowly lowered it so water would not overflow the sides of the mash tun, but it ran up through the false bottom so lowering took less than two minutes.  This is slower than the last time, and the grain bed is slightly lower, indicating the finer crush is settling tighter into the malt pipe.  I then started recirculation.  Recirculation is much faster, and the grain bed is very noticably less compacted.  The grain is level with the top of the malt pipe.  I am running the pipe just fast enough that the wort just drips over the edge of the top of the malt pipe.  This might work better if a mesh screen was placed over the top of the grain bed.
    \item[1120] Mash in Complete. Temp is 120.
    \item[1150] Raised temperature in mash tube to 150.
    \item[1215] Hit 150, started timer for 1 hour.
    \item[1315] Raised mash tube up. Starged sparging, targeting 12.0L of water for sparging over approximately an hour.  
    \item[1420] 12.0 Liters completed sparging, letting wort finish draining from grain bed for remaining hour.  I need to reduce the rate at which the sparge water is flowing and spread it out more over the grain bed on top.
    \item[1421] Setting temperature of kettle to 220 so temperature will be near boil at the end of sparge.
    \item[1425] Removed malt pipe and placed in a bucket once the wort level was at 5.5 gallons (6 gallons actual).  The malt pipe continued to drain about a half liter once placed in the disposal bucket.  Next brew should sparge with a bit less, maybe 15 liters.  \textbf{Pre-boil SG: 1.043}
    \item[1510] Boil started, started timer for 20 minutes.
    \item[1530] Added 1.5 oz German Hersbrucker Hops 2.0\% in hop spider.  Started Timer for 55 minutes.
    \item[1625] Added 1 teaspoon irish moss and .5 oz  German Hersbrucker Hops 2.0\% in hop spider. Started Timer for 10 minutes.
    \item[1635] Started circulating through the plate chiller to sanitize everything for remaining 5 minutes.  Heat is on full. Started timer for 5 minutes.
    \item[1640] Boil complete.  Turned off the pump, closed the valve on the mash tun to stop flow due to gravity, with the hose recylcing cooled wort back into the mash tun, opened the valve on the cooling water and and let the plate chiller work until the temperature dropped to 68.  (I should have let this drop to the target temperature of 64.) Moved the hose to the fermentation bucket, opened the valve on the mash tun and oxygen tank, and started the pump, transferring all the wort to the fermentation bucket.  Dropped in the black tilt sensor and innoculated with the yeast. Post-boil volume is 4.5 gallons (5 gallons actual).  Filled air lock, moved to fermentation chamber 2 and set the temperature for 64 degrees.
    \item[1845] Finished Cleaning.

    \item[Target OG:] 1.059
    \item[Target Temperature:] 64
    \item[Actual OG:] 1.055
    \item[Actual Temperature:] 70
\end{description}

