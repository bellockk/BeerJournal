%------------------------------------------------------------------------------
\def\todaysdate{20220618}
\newday{\todaysdate}\label{\todaysdate}

\newentry[Brew]{0930 Brew Day}

\begin{itemize}
    \item Mixed up a batch of sanitizer with 1 gallon of hot tap water and 1 tbsp of One Step.
    \item Assembled brew hoses and fixtures - using left side of plate chiller for wort
\end{itemize}

\textbf{Recipie}
\begin{description}
    \item[5000g] Wheat Malt (4)
    \item[2000g] Munich Malt (1)
    \item[2000g] Pilsner Malt (2)
    \item[1000g] Caramunich Malt (3)
\end{description}

Stepped Mash Schedule:
\begin{description}
    \item[120F] 30 Minutes
    \item[150F] 60 Minutes
\end{description}

Boil Additions:
\begin{description}
    \item [40g Hallertau Hersbrucker] -70 minutes
    \item [10g Hallertau Hersbrucker] -15 minutes
    \item [5g Irish Moss] -15 minutes
\end{description}

\begin{description}
    \item[0930] Start Heating water.  Filled to 6 gallon mark (6.5 gallon actual) so that lines can be rinsed and filled once heating is complete. Lowered the level in the mash tun to 5.0 gallon mark (5.5 gallon actual) on mash tun by pumping to the drain.  This rinsed out a bunch of flaky black substance from the plate chiller.  The 5 gallon mark on the mash tun equates to 5.5 gallons of water, set temperature to 130F.  Put 16.0 liters in the sparge water heater, and started heat at 80C.  
    \item[0945] Grinding grain.  I adjusted the mill by tightening up the gap just a bit.  The last brew missed final numbers by just a bit, so I am hoping that if I crush a little bit finer and keep everything else the same that I will hit the numbers exactly.
    \item[1005] Water is up to temp.
    \item[1055] Set mash tun to 120. Started Mash in.  I dropped the malt pipe into the mash tun without stirring.  I slowly lowered it so water would not overflow the sides of the mash tun, but it ran up through the false bottom so lowering took less than two minutes.  This is slower than the last time, and the grain bed is slightly lower, indicating the finer crush is settling tighter into the malt pipe.  I then started recirculation.  Recirculation is much faster, and the grain bed is very noticably less compacted.  The grain is level with the top of the malt pipe.  I am running the pipe just fast enough that the wort just drips over the edge of the top of the malt pipe.  This might work better if a mesh screen was placed over the top of the grain bed.
    \item[1100] Mash in Complete. Temp is 120.
    \item[1130] Raised temperature in mash tube to 150.
    \item[1145] Hit 150, started timer for 1 hour.
    \item[1245] Raised mash tube up. Starged sparging, targeting 12.0L of water for sparging over approximately an hour.  
    \item[1330] 12.0 Liters completed sparging, letting wort finish draining from grain bed for remaining hour. Setting temperature of kettle to 220 so temperature will be near boil at the end of sparge.
    \item[1420] Removed malt pipe and placed in a bucket once the wort level was at 5.8 gallons (6.2 gallons actual).  The malt pipe continued to drain about a half liter once placed in the disposal bucket.  \textbf{Pre-boil SG: 1.043}
    \item[1425] Boil started, started timer for 20 minutes.
    \item[1445] Added 1.5 oz German Hersbrucker Hops 2.0\% in hop spider.  Started Timer for 55 minutes.
    \item[1540] Added 1 teaspoon irish moss and .5 oz  German Hersbrucker Hops 2.0\% in hop spider. Started Timer for 10 minutes.
    \item[1550] Started circulating through the plate chiller to sanitize everything for remaining 5 minutes.  Heat is on full. Started timer for 5 minutes.
    \item[1555] Boil complete.  Turned off the pump, closed the valve on the mash tun to stop flow due to gravity, with the hose recylcing cooled wort back into the mash tun, opened the valve on the cooling water and and let the plate chiller work until the temperature dropped to 68.  (I should have let this drop to the target temperature of 64.) Moved the hose to the fermentation bucket, opened the valve on the mash tun and oxygen tank, and started the pump, transferring all the wort to the fermentation bucket.  Dropped in the blue tilt sensor and innoculated with the yeast. Post-boil volume is 4.5 gallons (5 gallons actual).  Filled air lock, moved to fermentation chamber 2 and set the temperature for 64 degrees.
    \item[1845] Finished Cleaning.

    \item[Target OG:] 1.059
    \item[Target Temperature:] 64
    \item[Actual OG:] 1.050
    \item[Actual Temperature:] 69
\end{description}

